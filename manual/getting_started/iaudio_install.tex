The \playername{} has a built-in bootloader which performs the
firmware update and can also access the hard drive via USB.  The
Rockbox bootloader can therefore be very minimalistic, as it does not require
 its own USB mode.  This makes it less dangerous to install the Rockbox bootloader
 as you can always restore it using the \playerman{} bootloader.

\note{The Rockbox bootloader overwrites the original firmware, making it
   impossible to dual-boot.}

\subsubsection{Installation}
\begin{itemize}
\item Download the Rockbox bootloader binary from 
\url{http://download.rockbox.org/bootloader/iaudio/}.
  \opt{iaudiox5}{Use the \fname{x5v\_fw.bin} file if your \dap{} is an X5V (without radio).
      If it is an X5 or X5L, use the \fname{x5\_fw.bin} file.}
  \opt{iaudiom5}{Use the \fname{m5\_fw.bin} file.}
  \opt{iaudiom3}{Use the \fname{cowon\_m3.bin} file.}
\item Copy it to the \fname{FIRMWARE} directory on your \dap{}.
\end{itemize}
